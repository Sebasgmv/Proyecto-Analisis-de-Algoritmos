\documentclass[10pt,twocolumn]{article}
\usepackage[utf8]{inputenc}
\usepackage{amsmath}
\usepackage{amsfonts}
\usepackage{amssymb}
\usepackage{graphicx}
\usepackage[left=2cm,right=2cm,top=2cm,bottom=2cm]{geometry}


\usepackage{listings}
\usepackage{color} %red, green, blue, yellow, cyan, magenta, black, white
\definecolor{mygreen}{RGB}{28,172,0} % color values Red, Green, Blue
\definecolor{mylilas}{RGB}{170,55,241}

\lstset{language=Matlab,%
	%basicstyle=\color{red},
	breaklines=true,%
	morekeywords={matlab2tikz},
	keywordstyle=\color{blue},%
	morekeywords=[2]{1}, keywordstyle=[2]{\color{black}},
	identifierstyle=\color{black},%
	stringstyle=\color{mylilas},
	commentstyle=\color{mygreen},%
	showstringspaces=false,%without this there will be a symbol in the places where there is a space
	%numbers=left,%
	%numberstyle={\tiny \color{black}},% size of the numbers
	numbersep=9pt, % this defines how far the numbers are from the text
	emph=[1]{for,end,break},emphstyle=[1]\color{red}, %some words to emphasise
	%emph=[2]{word1,word2}, emphstyle=[2]{style},    
}

\author{D.\\
	\small \textit{Universidad Pontificia Javeriana}\\
	\small \today}
\title{\textbf{Analisis de Algoritmos}}
\date{\vspace{-5ex}}

\begin{document}
	
	\twocolumn[
	\begin{@twocolumnfalse}
		\maketitle
		\begin{abstract}
			\begin{center}
			Resumen: Incluye un título descriptivo para tu artículo, seguido de un resumen breve y conciso del contenido que se va a tratar.
				
			\end{center}
		\end{abstract}
		KEYWORDS: \textit{\\ \\}
	\end{@twocolumnfalse}
	]
	
	\section{Introducción}
	
	 En esta sección, describe el problema que se va a abordar en el artículo y su importancia en la teoría de la complejidad y análisis de algoritmos. También es importante incluir un poco de contexto histórico y la relevancia del problema que se va a tratar.
	
	\section{Marco Teorico}
	
	En esta sección, describe los conceptos teóricos que se utilizarán para el análisis de los algoritmos y las estructuras de datos empleadas. Además, incluye referencias a los trabajos relevantes previos en el área.
	

	
	\section{Descripcion del problema}
	
	En esta sección, describe el juego sencillo que se va a utilizar como caso de estudio, incluyendo una descripción de las reglas del juego, los objetivos y los posibles desafíos que deben ser superados.
	
	
	\section{Analisis de algoritmos}
	
	En esta sección, describe los algoritmos que se van a analizar y sus características. Es importante incluir el pseudocódigo de los algoritmos, su complejidad temporal y espacial, y una justificación de por qué se seleccionaron.
	
	\section{Implementacion y resultados}
	
	En esta sección, describe cómo se implementaron los algoritmos en el juego sencillo y los resultados obtenidos. Incluye tablas y gráficas para mostrar los resultados de la implementación y comparaciones entre los algoritmos.
	
	\section{Discusion y conclusiones}
	
	En esta sección, discute los resultados obtenidos, la eficacia de los algoritmos en el contexto del problema, y cualquier limitación o aspecto a mejorar. También es importante incluir posibles aplicaciones futuras y las lecciones aprendidas en la implementación de los algoritmos.
	
		
\bibliographystyle{ieeetr}
\bibliography{mybib}
	
	
\end{document}